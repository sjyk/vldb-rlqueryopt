\section{Background}
First, we will introduce the algorithmic connection between Reinforcement Learning and the join optimization problem.

\subsection{Query Model}
Consider the following query model.
Let $\{R_1,...,R_T\}$ define a set of relations, and let $\mathcal{R} = R_1 \times ... \times R_T$ denote the cartesian product.
We define an inner join query $q$ as a subset of $q \subseteq \mathcal{R}$ as defined by a conjunctive predicate $\rho_1 \wedge ... \wedge \rho_j$ where each expression $\rho$ is binary boolean function of the attributes of two relations in the set $\rho = R_i.a \{=,!=,>,<\} R_j.b)$. We assume that the number of conjunctive expressions is capped at some fixed size $\mathcal{N}$.
We further assume that the join operation is commutative, left associative, and right associative.
Extending our work to consider left and right outer joins is very straight-forward but we will defer that to that to future work.

We will use the following database of three relations denoting employee salries as a running example throughout the paper:
\[
\text{Emp}(id, name, rank)
\]
\[
\text{Pos}(rank, title, code)
\]
\[
\text{Sal}(code, amount)
\]
Consider the following join query:
\begin{lstlisting}
SELECT *
 FROM Emp, Pos, Sal
 WHERE Emp.rank = Pos.rank AND
 Pos.code = Sal.code
\end{lstlisting}
In this schema, $\mathcal{R}$ denotes the set of tuples $\{(e \in Emp, p \in Pos, s \in Sal)\}$. There are two predicates $\rho_1 = Emp.rank = Pos.rank$ and $\rho_2 = Pos.code = Sal.code$, combined with a conjunction. 
One has several possible options on how to execute that query. For example, one could execute the query as $Emp \bowtie (Sal \bowtie Pos)$. Or, one could execute the query as $Sal \bowtie (Emp \bowtie Pos)$. 

\subsection{Graph Model of Enumeration}
Enumerating the set of all possible dyadic join plans can be expressed as operations on a graph representing the join relationships requested by a query.

\begin{definition}[Query Graph]
Let $G$ define an undirected graph called the \emph{query graph}, where each relation $R$ is a vertex and each $\rho$ defines an edge between vertices. The number of connected components of $G$ are denoted by $\kappa_G$.
\end{definition}

Each possible dyadic join is equivalent to a combinatorial operation called a graph contraction.

\begin{definition}[Contraction]
Let $G = (V,E)$ be a query graph with V defining the set of relations and E defining the edges from the join predicates. A contraction $c$ is a function of the graph parametrized by a tuple of vertices $c=(v_i, v_j)$. Applying $c$ to the graph $G$ defines a new graph with the following properties: (1) $v_i$ and $v_j$ are removed from $V$, (2) a new vertex $v_{ij}$ is added to $V$, and (3) the edges of $v_{ij}$ are the union of the edges incident to $v_i$ and $v_j$. 
\end{definition}

Each contraction reduces the number of vertices by $1$. And, that every feasible dyadic join plan can be described as a sequence of such contractions $c_1 \circ c_2 ...\circ c_{T}$ until $|V| = \kappa_G$. Going back to our running example, suppose we start with a query graph consisting of the vertices $(Emp, Pos, Sal)$. Let the first contraction be $c_1 = (Emp, Pos)$, this leads to a query graph where the new vertices are $(Emp+Pos, Sal)$. Applying the only remaining possible contraction, we arrive at a single remaining vertex $Sal+(Emp+Pos)$ corresponding to the join plan $Sal \bowtie (Emp \bowtie Pos)$. 

The join optimization is to find the best possible one of these contraction sequences. Assume that we have access to a cost model $J$, which is a function that can estimate the incremental cost of a particular contraction $J(c) \mapsto \mathbb{R}_+$. 

\begin{problem}[Join Optimization Problem]
Let $G$ define a query graph and $J$ define a cost model. Find a sequence of $c_1 \circ c_2 ...\circ c_{T}$ terminating in $|V| = \kappa_G$ to minimize:
\[
\min_{c_1,...,c_T} \sum_{i=1}^T J(c_i)
\]\label{joinopt}
\end{problem}

Note how the ``Principle of Optimality'' arises in Problem \ref{joinopt}. Applying each contraction generates a subproblem of a smaller size (a query graph with one less vertex). Each of these subproblems must be solved optimally in any optimal solution. Also note that this model can capture physical operator selection as well. The set of allowed contractions can be typed with an eligible join type e.g., $c=(v_i, v_j, \text{hash})$ or $c=(v_i, v_j, \text{index})$.

\subsection{Greedy vs. Optimal}
A \emph{greedy} solution to this problem is to optimize each $c_i$ independently. The algorithm proceeds as follows: (1) start with the query graph, (2) find the lowest cost contraction, (3) update the query graph and repeat until only one vertex is left. The greedy algorithm, of course, does not consider how local decision might affect future costs. Consider our running example query with the following costs (assume symmetry):
\[J(EP)= 100,~J(SP)= 90,~J((EP)S)= 10,~J((SP)E)= 50\]
The greedy solution would result in a cost of 140 (because it neglects the future effects of a decision), while the optimal solution has a cost of 110.
However, there is an upside, this greedy algorithm has a computational complexity of $O(|V|^3)$---despite the super-exponential search space.

The decision at each index needs to consider the long-term value of its actions where one might have to sacrifice a short term benefit for a long term payoff.
Consider the following way of expressing the optimization problem in the problem statement for a particular query graph $G$:
\begin{equation}
V(G) = \min_{c_1,...,c_T} \sum_{i=1}^T J(c_i)
\label{eq:main}
\end{equation}

We can concisely describe Equation \ref{eq:main} as the function $V(G)$, i.e., given the initial graph $G$, what is the value of acting optimally till the end of the decision horizon. This function is conveniently named the \emph{value function}. Bellman's ``Principle of Optimality'' noted that optimal behavior over an entire decision horizon implies optimal behavior from any starting index $t>1$ as well, which is the basis for the idea of dynamic programming.
So, $V(G)$ can be then defined recursively for any subsequent graph $G'$ generated by future contractions:
\begin{equation}
V(G) = \min_{c}\{~J(c) + \gamma \cdot V(G') ~\}
\label{eq:value}
\end{equation}
Usually, write this value recursion in the following form:
\[
Q(G,c) = J(c) + \gamma \cdot V(G')
\]
Leading to the following recursive definition of the Q-Function (or cost-to-go function):
\begin{equation}
Q(G,c) = J(c) + \gamma \cdot \min_{c'} Q( G',c')
\label{eq:q}
\end{equation}

The Q-Function describes the long-term value of each contraction. 
That means at each $G'$ local optimization of $\min_{c'} Q(G',c')$ is sufficient to derive an optimal sequence of decisions. Put another way, the Q-function is a hypothetical cost function on which greedy descent is optimal and equivalent to solving the original problem.
If we revisit the greedy algorithm, and revise it as follows: (1) start with the query graph, (2) find the lowest \textbf{Q-value} contraction, (3) update the query graph and repeat. This algorithm has a computational complexity of $O(|V|^3)$ (just like the greedy algorithm) but is provably optimal. 

The Q-function is implicitly stored as a table in classical dynamic programming approaches, such as the System R enumeration algorithm. In the System R algorithm, there is a hash table that maps sets of joined relations to their lowest-cost access path:
\begin{lstlisting}
HashMap<Set<Relation>, (Plan, Long)> bestJoins;
\end{lstlisting}
As the algorithm enumerates more subplans, if a particular relation set exists in the table it replaces the access path if the enumerated plan has a lower cost than one in the table. We could equivalently write a different hash table that moves around the elements on the table described above. Instead of a set of relations mapping to a plan and a cost, we could consider a set of relations \emph{and} a plan mapping to a cost: 
\begin{lstlisting}
HashMap<(Set<Relation>, Plan), Long> bestQJoins;
\end{lstlisting}
As the algorithm enumerates more plans, the algorithm put enumerated plans into the table. However, the replacement policy for this table is slightly different than in the classical model. As we enumerate plans, we replace all child plans with the \emph{cost of the parent} if lower. Once this table is constructed, backtracking can be used to retrieve an optimal plan.

\subsection{Learning the Q-Function}
This change would be less efficient implementation in classical dynamic programming, but it crucially allows us to setup a machine learning problem. 
What if we could regress from features of \texttt{(Set<Relation>, Plan)} to a cost based on a small number of observations?
Instead of a Q-Function as a table, it is parametrized as a model:
\[
Q_\theta(f_G,f_c) \approx Q(G,c)
\]
where $f_G$ is a feature vector representing the query graph and $f_c$ is a feature vector representing the particular contraction on the graph. $\theta$ defines the neural network parameters that represent this function. 

The basic strategy to generate observational data, that is, real executions of join plans, and optimize $\theta$ to best explain the observations. Such an optimization problem is a key motivation of a general class of algorithms called Reinforcement Learning~\cite{sutton1998reinforcement}, where statistical machine learning techniques are used to approximate optimal behavior while observing substantially less data that full enumeration. 

For those familiar with the AI literature, this problem defines a Markov Decision Process. $G$ is exactly a representation of the \textbf{state} and $c$ is a representation of the \textbf{action}.
The utility function (or reward) of this process is the negative overall runtime.
The objective of the planning problem is to find a policy, which is a map from a query graph to the best possible join $c$.

In the popular Q-Learning approach~\cite{sutton1998reinforcement}, the algorithm enumerates random samples of decision sequences containing $(G,c, runtime, G')$ tuples forming a trajectory. From these tuples, one can calculate the following value:
\[
y_i = runtime + \arg \max_{u} Q_\theta(G',c)
\]
Each of the $y_i$ can be used to define a loss function since if $Q$ were the true Q function, then the following recurrence would hold:
\[
Q(G,c) = runtime + \arg \max_{u} Q_\theta(G',c)
\]
So, Q-Learning defines a loss:
\[
L(Q) = \sum_{i} \|y_i - Q_\theta(G',c)\|_2^2
\]
This loss can be optimized with gradient descent:
\[
\theta^{(i+1)} = \theta^{(i)} + \alpha \cdot [ (R + \max_{c'} Q_\theta(G',c')) - Q_\theta(G,c) ] \cdot \nabla Q_\theta(G,c) 
\]



This algorithm is called the Deep Q Network algorithm~\cite{mnih2015human} and was used to learn how to autonomously play Atari Games.
The key implication is that the neural network allows for some ability for the optimizer to extroplate the cost-to-go even for plans that are not enumerated. This means that if the featurization $f_G$ and $f_c$ are designed in a sufficiently general way, then the neural network can represent cost-to-go estimates across an entire workload--not just a single planning instance.

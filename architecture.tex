\section{Optimizer Architecture}
We call this new join optimization algorithm \sys, and describe how it fits into a query optimization stack. \sys optimizes Select-Project-Join (SPJ) blocks of SQL queries.

\subsection{Architecture and API}
First, we describe the information that \sys requires to train, learn, and execute.
In this paper, we consider deployment on a fixed database.
The algorithm has a training phase in which it observes join costs and an execution phase where it evaluates the learned model.
In principle, one could consider a more adaptive setting where the model is periodically retrained. We will defer this discussion to future work.

We envision that \sys fits into an existing optimizer stack as a replacement for the join optimization component.
\sys makes relatively minimal assumptions about the structure of the optimizer. Below are the API hooks that we require implemented:

\vspace{0.5em}
\begin{lstlisting}
sample(): List<Queries> 
\end{lstlisting}
\noindent \emph{Workload Generation.} A function that returns a list of training queries of interest. \sys requires a relevant workload for training. In our experiments, we show that this workload can be taken from query templates or sampled from the database schema.


\vspace{0.5em}
\begin{lstlisting}
train(query): List<Graph,Join,Graph',Cost> 
\end{lstlisting}
\noindent \emph{Workload Generation.} A function that given a query returns a list of join actions and their resultant costs. \sys requires the system to have its own planner to generate training data. This means generating feasible join plans and their associated costs. Our experiments evaluate integration with both deterministic enumeration algorithms (bushy and left-deep) and a randomized algorithm called QuickPick-1000.


\vspace{0.5em}
\begin{lstlisting}
selectivity(predicate): Double
\end{lstlisting}
\noindent A function that returns the reduction factor of a particular single table predicate. \sys leverages the optimizer's own selectivity estimate for featurization. This captures the effects of single table selections on plan costs.

\subsection{Implementation}
Our prototype implementation is integrated with Apache Calcite~\cite{calcite}.
Apache Calcite provides libraries for parsing SQL, representing relational algebraic expressions, and a Volcano-based query optimizer~\cite{graefe1991volcano}. Calcite does not handle physical execution or storage and uses JDBC connectors to a variety of database engines and file formats.
We implemented a package inside Calcite that allowed us to leverage its parsing and plan representation, but also augment it with more sophisticated cost models and optimization algorithms.
We implemented an internal simulator to simulate cost estimates based on histograms, true cardinalities, and other approximations.
All of our code is written in single-threaded Java.

\subsection{Architectural Implications}
\textbf{TODO: Joe wanted to write something here}


\subsection{Limitations}
This work is meant to be an initial study of Deep Reinforcement Learning in the context of join optimization. There are a number of features expected of join optimizers that are not yet supported. We plan to add such support in future work but felt it was out of scope of this paper.

 \vspace{0.5em} \noindent \textbf{Sort Orders: } We do not consider cost models for sorting, sort-merge joins, or interesting orders. Keeping track of interesting sort orders in subplans is actually not difficult in our framework. They can be tracked as a set of additional features. 

\vspace{0.5em} \noindent  \textbf{Outer Joins/Inequality Joins: } We currently only consider optimization of inner joins and conjunctions of join conditions.


\section{Optimizer Architecture}
In the previous section, we described the join enumeration algorithm. Now, we contextualize the enumeration algorithm in a more fully featured optimization stack. In particular, there are aspects of the featurization and graph definitions that have to change based on the nature of the queries.

\subsection{Selections and Projections}
To handle single relation selections and projections in the query, we have to tweak the feature representation. This is because upstream selections and projections will change the cost properties of downstream joins. As in the classical optimizers, we eagerly apply selections and projections to each relation. 

This means that each relation in the query graph potentially a different number of attributes and a different cardinality than the base relation. While the proposed featurization does capture the visible attributes, it does not capture changes in cardinality due to upstream selections. 

Here, we leverage the table statistics present in most RDBMS. For each relation in the query graph, we can estimate a reduction factor $\delta_{r}$, which is an estimate of the fraction of tuples present after applying the selection to relation $r$. 
In the featurization described in the previous section, we have a set of binary features $V_{rel}$ that describe the participating attributes.
We multiply the reduction factors $\delta_r$ for each table with the features corresponding to attributes derived from that relation.

\subsection{Physical Operator Selection}
To add support for an optimizer that selects physical operators, we simply have to add ``labeled'' contractions, where certain physical operators are eligible. Suppose we have a set of possible physical operators, e.g., \textsf{nestedLoop}, \textsf{sortMerge}, \textsf{indexLoop}. We would simply add an additional feature to the Q-Function that captures which physical operator is selected.

\subsection{Indexes and Sort Orders}


\subsection{Summary and System Architecture}
Surprisingly, we found that we could build a relatively full featured query optimizer based on a Deep RL join enumeration strategy.
Our system, called \qsys, is built on Apache Calcite.
The system connects to various database engines through a JDBC connector. It executes logically optimized queries by re-writing SQL expressions and setting hints.
The system parses standard SQL and the learning steps are implemented using \textsf{DL4J}.
The optimizer has two modes, training and execution. In the training mode, the optimizer will collect data and based on a user set parameter execute suboptimal plans. In the execution mode, the optimizer will leverage the trained model to improve its optimizer performance.

